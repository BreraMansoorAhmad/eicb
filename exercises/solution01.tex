\documentclass[
  ngerman,
  DIV=14
]{scrartcl}
\usepackage{babel}
\usepackage{csquotes}

% typography
\usepackage{fontspec}
\usepackage[utopia]{mathdesign}
\setsansfont{Open Sans}[
  BoldFont={Open Sans Bold},
  ItalicFont={Open Sans Italic}]
\setmonofont[Scale=0.85]{Menlo}
\setmainfont{Palatino}
\linespread{1.15}
%\renewcommand\familydefault{\sfdefault}
\usepackage[factor=2000]{microtype}

% graphics, drawings, etc.
\usepackage{xcolor}
\usepackage{graphicx}
\usepackage[most]{tcolorbox}
\usepackage{tikz}
\usetikzlibrary{shapes.geometric}
\usetikzlibrary{shapes.arrows}
\usetikzlibrary{positioning}
\newtcolorbox{anmerkung}{%
  grow to left by=10pt,
  colback=black!10,
  colframe=white,
  coltitle=black,
  borderline west={4pt}{0pt}{black!30},
  boxrule=0pt,
  boxsep=0pt,
  %breakable,
  enhanced jigsaw,
  title={Anmerkung\par},
  fonttitle={\bfseries},
  attach title to upper={}}
\newtcolorbox{hinweis}{%
  grow to left by=10pt,
  colback=black!10,
  colframe=white,
  coltitle=black,
  borderline west={4pt}{0pt}{black!30},
  boxrule=0pt,
  boxsep=0pt,
  %breakable,
  enhanced jigsaw,
  title={Hinweis\par},
  fonttitle={\bfseries},
  attach title to upper={}}

% highlighting, lists, code
\usepackage{soul}
\usepackage{enumitem}
\usepackage{listings}
\lstset{
  basicstyle=\ttfamily,
  %escapeinside=||,
  keywordstyle=\color{blue!50!black},
  stringstyle=\color{green!50!black}}
\capsdef{////}{\scshape}{.16em}{.4em}{.2em}

% math
\usepackage{amsmath}

% nice tables
\usepackage{booktabs}
\newcommand{\tablespacing}[1]{\renewcommand{\arraystretch}{#1}}

% links
\usepackage[
  colorlinks,
  linkcolor={red!50!black},
  citecolor={blue!50!black},
  urlcolor={blue!80!black}
]{hyperref}

\title{Lösungsblatt Nr. 1}
\date{Wintersemester 2018-2019}
\author{Andreas Koch}
\subject{Einführung in den Compilerbau}
\subtitle{von Patrick Elsen, Viola und Michael Matthé}
\publishers{Technische Universität Darmstadt}

\begin{document}
\maketitle

\subsection*{Einleitung}

Auf diesem Aufgabenblatt sollen Sie sich mit der Matrix and Vector Language, kurz MAVL, vertraut machen. Studieren Sie bitte zunächst die MAVL-Sprachspezifikation, die Sie im Moodle-Kurs der Veranstaltung finden.

\subsection*{Aufgabe 1.1: MAVL-Syntax}

Die MAVL-Sprachspezifikation enthält nur eine informelle Beschreibung der Syntaxelemente der Sprache. In den folgenden Teilaufgaben sollen Sie einige der Syntaxelemente in Produktionen einer kontextfreien Grammatik überführen.

\subsection*{Aufgabe 1.1a: Produktionen}

Geben Sie Produktionen für die Nichtterminale \texttt{mulExpr} (Multiplikations-Operator), \texttt{subvectorExpr} (Subvektor-Operator), sowie \texttt{recordElementSelectExpr} (Selektion von Record-Elementen) an.

\bigskip\noindent
Ein Multiplikationsausdruck ist in der Sprachspezifikation unter \emph{§~7.5 Ternärer Operator} definiert. Ein solcher Ausdruck nimmt \verb|int| oder \verb|float|-Werte als Parameter und ist Linksassoziativ. Also kann man diesen grammatikalisch folgendermaßen definieren.
\begin{verbatim}
    mulExpr ::= (INT | FLOAT) '*' expr
\end{verbatim}
Die \verb|subvectorExpr| ist in dem Sprachstandard unter \emph{§~7.7.4 Submatrix und Subvektor} definiert. Hier wird definiert, dass eine solche Beispielsweise als \verb|v{-1:i:1}| geschrieben werden kann, wobei \verb|v| ein Vektor und \verb|i| eine Zahl sein muss. Dieser Ausdruck extrahiert einen Subvektor mit den Elementen $[i-1, i+1]$. 
\begin{verbatim}
    subvectorExpr ::= '{' expr ':' expr ':' expr '}'  
\end{verbatim}
Unter \emph{§~7.8 Selektion von Record-Elementen} ist definiert, wie der Syntax funktioniert.
\begin{verbatim}
    recordElementSelectExpr ::= ID '@' ID  
\end{verbatim}

\subsection*{Aufgabe 1.1b}

Geben Sie Produktionen für die Nichtterminale \verb|primitiveType| (primitive Typen) und \verb|vectorType| (Vektortypen) an.

\bigskip\noindent
Die primitiven Typen sind unter \emph{§~4.2 Primitive Datentypen} definiert. Hier sind nur \texttt{int}, \texttt{float} und \texttt{bool} als eingebaute, primitive Typen angegeben. Also könnte eine Grammatik folgendermaßen aussehen:
\begin{verbatim}
    primitiveType ::= 'int' | 'float' | 'bool'
\end{verbatim}
Der \texttt{vectorType} ist bei \emph{§~4.5 Vektoren} definiert. Ein Vektor muss, mit einem Elementtyp (entweder \texttt{int} oder \texttt{float}) und einer Länge (positive, ganze Zahl) definiert werden.
\begin{verbatim}
    vectorType ::= 'vector' '<' ('int' | 'float') '>' '[' constExpr ']'
\end{verbatim}

\subsection*{Aufgabe 1.1c}

Geben Sie Produktionen für die Nichtterminale \texttt{returnStmt} (Rückgabebefehl), \texttt{varDecl} Variablendeklaration), \texttt{callStmt} (Aufruf-Befehl, ohne Rückgabewert) sowie \texttt{forStmt} (For-Schleife) an.




\subsection*{Aufgabe 1.2: AST zu MAVL}
Abstrakte Syntaxbäume (engl. \emph{Abstract Syntax Trees, AST}) sind eine weitverbreitete Zwischendarstellung, die nur essentielle Informationen enthält und Details der konkreten Syntax einer Programmiersprache abstrahiert.

\medskip\noindent
In dieser Aufgabe zeigen wie Ihnen eine mögliche Repräsentation von MAVL-Code als AST. Die darin verwendeten AST-Knoten korrespondieren auf natürliche Weise mit den in der Spezifikation beschriebenen Syntaxelementen. 

\subsection*{Artikel 1.2a}
Geben Sie den zum folgenden AST zugehörigen MAVL-Code an.  

\bigskip\noindent
Den Syntaxbaum kann man, von oben nach unten und links nach rechts, einfach wie Code lesen.
\begin{lstlisting}
if(id && r > q) {
  r = -1;
  q(q, r);
} else {
  r = a - (q * d);
}
\end{lstlisting}


\subsection*{Artikel 1.2b}
Geben Sie den zum folgenden AST zugehörigen MAVL-Code an.  

\bigskip\noindent
Antwort.

\subsection*{Artikel 1.3: Ausdrücke}
Ausdrücke in typischen Programmiersprachen lassen sich einfach durch mehrdeutige Grammatiken beschreiben, die aber als Grundlage für die syntaktische Analyse ungeeignet sind.

\subsection*{Artikel 1.3a}
Zeichnen Sie den AST für den MAVL-Ausdruck \texttt{q \& a == b \# c}.


\bigskip\noindent
Antwort.

\subsection*{Artikel 1.3b}
Zeichnen Sie den AST für den MAVL-Ausdruck \texttt{- v2 .* v1 + (v2 \# m)[0]}.


\bigskip\noindent
Antwort.

\subsection*{Aufgabe 1.3c}
Gegeben seien folgende Wertedefinitionen.

Welchen Wert liefert der Ausdruck aus Teilaufgabe 1.3b?

\bigskip\noindent
Antwort.

\end{document}