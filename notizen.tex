\documentclass{scrartcl}
\usepackage{fontspec}
\setmainfont{Open Sans}[
  BoldFont={Open Sans Bold},
  ItalicFont={Open Sans Italic}]
\setsansfont{Open Sans}[
  BoldFont={Open Sans Bold},
  ItalicFont={Open Sans Italic}]
\setmonofont{Menlo}
\usepackage{xcolor}
\usepackage{soul}
\usepackage{enumitem}

\title{Einführung in den Compilerbau}
\date{Wintersemester 2018-2019}
\author{Andreas Koch}

\begin{document}
  \maketitle
  \tableofcontents
  \newpage
  
  %\KOMAoptions{twocolumn}
  
  \section{Organisatorisches}
   
  \subsection{Grundlage der Vorlesung}
   
  Die Vorlesung basiert \hl{fast vollständig} auf \emph{Programming Language Processors in Java}\footnote{von David Watt und Deryck Brown, Prentice-Hall 2000}. Auszugsweise noch weiteres Material, z.\ B.\ zum ANTLR-Parsergenerator.
  
  \subsection{Übersichtswerk}
  
  Einen guten allgemeinen Überblick, aber im Detail mit anderen Schwerpunkten als diese Vorlesung, bietet \emph{Compilers, 2. Auflage}\footnote{Von Aho, Sethi, Ullmann, Lam, Addison-Wesley 2006. Auch auf Deutsch verfügbar.}.
   
  \subsection{Aufbau der Veranstaltung}
  
  Diese Veranstaltung ist logisch in mehrere Teile gegliedert.
  
  \begin{itemize}
    \item Überblick über Front-End (ca.\ 3 Wochen):
    \begin{itemize}
      \item Lexing und Parsing,
      \item Zwischendarstellungen.
    \end{itemize}
    \item Überblick über Middle-End (ca.\ 2 Wochen):
    \begin{itemize}
      \item Semantische- und Kontextanalyse.
    \end{itemize}
    \item Übersicht über Back-End (ca.\ 4 Wochen):
    \begin{itemize}
      \item Laufzeitorganisation,
      \item Code-Erzeugung.
    \end{itemize}
    \item Verwendung von Front-End-Generatoren (ca.\ 2--3 Wochen),
    \item Java Virtuelle Maschine (ca.\ 1--2 Wochen).
  \end{itemize}

  Die ersten drei Teile der Veranstaltung richten sich an die Veranstaltungen:
  
  \begin{itemize}
    \item IMT3052 von Ivar Farup, Universität Grøvik, Norwegen,
    \item Vertalerbouw von Theu Ruys, Universität Twente, Niederlande.
  \end{itemize}
  
  

\end{document}
